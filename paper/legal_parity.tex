
\section{Legal Parity}

	In order to properly reproduce the same \textit{functionality} of a paper-based document electronically, we must  Functional equivalence of possession is achieved when a reliable method is employed to establish control of that record by a person and to identify the person in control. 

  We make a distinction between \textit{functionality} and \textit{legal parity}. That is to say with \textit{functionality}, there may exist an authoratiative piece of physicallity \textit{(i.e. a physical piece of paper)}, where as our system achieves full \textit{legal parity} through a novel, patent pending process described in Appendix B. 

  \subsection{Control and Physicallity}

	The notion of \textit{control} when used as a substitute for possession requires a reliable method for identifying the current party in control of a specific electronic record as the said notion typically focuses on the identity of the person entitled to enforce the rights embodied in the electronic transferable record.

  The method of identification may be accomplished through a closed system, or through an open system. Under the draft model law, the notion of original and uniqueness has been connected to control. 

  Emphasis has been given to reliably ensure that the claim may be presented to the debtor only once. 

	For example. an indicative list of transferable documents or instruments includes: bills of exchange, cheques, promissory notes, consignment notes, bills of lading, warehouse receipts, cargo insurance certificates and air waybills.

  Under this \textit{protocol}, we use the \textbold{legal entity identifier} to record issuances against.
  \footnote{Legal Entity Identifier Golden Record, https://www.gleif.org/en/lei-data/gleif-golden-copy/download-the-golden-copy#/}

\subsection{Negotiable Bills of Lading}
A common carrier issuing a negotiable bill of lading has a lien on the goods covered by the bill for—

1. Charges for storage, transportation, and delivery including demurrage and terminal charges,, and expenses necessary to preserve the goods or incidental to transporting the goods after the date of the bill; and

\(2\) other charges for which the bill expressly specifies a lien is claimed to the extent the charges are allowed by law and the agreement between the consignor and carrier.

\textbf{ Negotiable Bills}

(1) A bill of lading is negotiable if the bill
  (A) states that the goods are to be delivered to the order of a consignee; and
  (B) does not contain on its face an agreement with the shipper that the bill is not negotiable.

(2) Inserting in a negotiable bill of lading the name of a person to be notified of the arrival of the goods—

  (A) does not limit its negotiability; and
  (B) is not notice to the purchaser of the goods of a right the named person has to the goods.

 \textbf{Nonnegotiable Bills}

  (1) A bill of lading is nonnegotiable if the bill states that the goods are to be delivered to a consignee. The endorsement of a nonnegotiable bill does not—

    (A) make the bill negotiable; or
    (B) give the transferee any additional right.

  (2) A common carrier issuing a nonnegotiable bill of lading must put “nonnegotiable” or “not negotiable” on the bill. This paragraph does not apply to an informal memorandum or acknowledgment.

\footnote{Pub. L. 103–272, § 1(e), July 5, 1994, 108 Stat. 1346.}